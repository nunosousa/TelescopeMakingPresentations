\documentclass{beamer}
\usepackage[utf8]{inputenc}
\usepackage{hyperref}

\usetheme{Madrid}
\usecolortheme{beaver}
\hypersetup{colorlinks=true, linkcolor=blue, filecolor=magenta, urlcolor=cyan}
\urlstyle{same}

\title{Amateur Telescope Making}
\subtitle{...or how to make your own telescope at home}
\author{Nuno Miguel Sousa}
\institute{Coimbra, Portugal}
\date{\today}

\begin{document}

\begin{frame}
\titlepage
\end{frame}

\begin{frame}
\frametitle{What is Amateur Telescope Making?}
\begin{block}{}
Amateur Telescope Making, or ATM, is a hobby taken by people that have an interest in astronomic observation and enjoy building telescopes.\footnotemark
\end{block}
\footnotetext{\url{https://en.wikipedia.org/wiki/Amateur_telescope_making}}
\begin{block}{}
ATM can range from just assembling the individually bought components to actually fabricate some or all of the components of a telescope.
\end{block}
\begin{block}{}
The most common type of telescope made by hobbyists is the so called Newtonian reflector (invented by Sir Isaac Newton).\footnotemark
\end{block}
\footnotetext{\url{https://en.wikipedia.org/wiki/Newtonian_telescope}}
\end{frame}

\begin{frame}
\frametitle{Why build a telescope?}
\begin{itemize}
\item As with many hobbi
It is an attainable goal to produce by hand excellent quality optics, better than commercial optics on low end telescopes.
\end{itemize}
\end{frame}

\begin{frame}
\frametitle{My personal motivation for ATM}
I didn't have any significant exposure to astronomy in particular but found it an interesting subject nonetheless.
The tipping point to turn my attention to the night sky, was when by chance I had the opportunity to look up at the night sky in the middle of rural Alentejo in southern Portugal. The combination of very low light pollution and clear sky provided for a clear and very distinctive view of the Milky Way.
\end{frame}

\begin{frame}
\frametitle{About the author}
My name is Nuno and ...
\end{frame}

\begin{frame}
\frametitle{The end}
Thank you!
\end{frame}

\end{document}
