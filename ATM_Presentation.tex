\documentclass{beamer}
\usepackage[utf8]{inputenc}
\usepackage{hyperref}

\usetheme{Madrid}
\usecolortheme{beaver}
\hypersetup{colorlinks=true, linkcolor=blue, filecolor=magenta, urlcolor=cyan}
\urlstyle{same}

\title{Amateur Telescope Making}
\subtitle{...or how to make your own telescope at home}
\author{Nuno Miguel Sousa}
\institute{Coimbra, Portugal}
\date{\today}

\begin{document}

\begin{frame}
\titlepage
\end{frame}

\begin{frame}
\frametitle{What is Amateur Telescope Making?}
\begin{block}{}
Amateur Telescope Making, or ATM, is a hobby taken by people that have an interest in astronomic observation and enjoy building telescopes.\footnotemark
\end{block}
\footnotetext{\url{https://en.wikipedia.org/wiki/Amateur_telescope_making}}
\begin{block}{}
ATM can range from just assembling the individualy bought components to actually fabricate some or all of the components of a telescope.
\end{block}
\begin{block}{}
The most common type of telescope made by hobbyists is the so called Newtonian reflector (invented by Sir Isaac Newton).\footnotemark
\end{block}
\footnotetext{\url{https://en.wikipedia.org/wiki/Newtonian_telescope}}
\end{frame}

\begin{frame}
\frametitle{Why build a telescope?}
\begin{block}
It is an attainable goal to produce by hand excelent quality optics, better than commercial optics on low end telescopes.\footnotemark
\end{block}
\footnotetext{\url{https://skyandtelescope.org/astronomy-resources/astronomy-questions-answers/are-machine-made-telescope-mirrors-better-than-those-made-my-hand/}}
\end{frame}

\begin{frame}
\frametitle{My personal motivation for ATM}
I always liked space and astronomy, but the tipping point to have a telescope myself was when once I was in the middle of rural alentejo late at night and had the first look at the milky way without any urban light pollution.
\end{frame}

\begin{frame}
\frametitle{About the author}
My name is Nuno and ...
\end{frame}

\begin{frame}
\frametitle{The end}
Thank you!
\end{frame}

\end{document}
